\documentclass{article}
\usepackage{scimisc-cv}

\title{Ritika Shrivastava}
\author{Ritika Shrivastava}
\date{July 2020}

%% These are custom commands defined in scimisc-cv.sty
\cvname{Ritika Shrivastava}
\cvpersonalinfo{
760-442-4743 \cvinfosep
linkedin.com/in/ritikashrivastava/ \cvinfosep
ritishri@berkeley.edu
}

\begin{document}

% \maketitle %% This is LaTeX's default title constructed from \title,\author,\date

\makecvtitle %% This is a custom command constructing the CV title from \cvname, \cvpersonalinfo

\section{Education}
\cvsubsection{University of California, Berkeley} [August 2018 - May 2021]
[B.S. Electrical Engineering and Computer Science][Regents and Chancellor's Scholar (Top 2\% of incoming class.)]
\begin{itemize}
\item \textbf{Computer Science:} Data Structures, Discrete Mathematics, Algorithms, Artificial Intelligence, Computer Architecture
\item \textbf{Electrical Engineering:} Designing Information Devices and Systems I \& II, Robotics I, PCB design
\item \textbf{Statistics:} Probability and Random Processes
\end{itemize}

 \cvsubsection{University of San Diego, GenCyber Scholarship} [May 2017 - June 2017]
[Sponsored by National Security Agency and National Science Foundation][]
\begin{itemize}
\item Manipulated Network Security levels through process isolation and domain separation
\item Assessed risks and threats to a system using least Privilege and modularity techniques
\item Ran forensics and applied cryptography techniques (Abstraction and Layering)
\end{itemize}
 
\section{Skills}
\begin{description}[widest=Langauges]
\item[Languages] Python, GoLang, Java, C, Scheme, SQL, RISC-V, HTML, CSS, JavaScript
\item[Python Libraries] ROS, Pandas, NumPy, OpenCV, sklearn, SciPy, TensorFlow
\item[Security Tools \& APIs] Shodan.io, Censys, VirusTotal, Anomali, Nmap, Wireshark, Cuckoo 
\end{description}

\section{Work Experience}

%% Another custom command provide by scimisc-cv.sty.
%% First two argumetns are typeset on the first line in bold; 3rd and 4th arguments are typset on second line in italics. 2nd, 3rd and 4th arguments are OPTIONAL
\cvsubsection{Microsoft: Defender ATP - DeepResearch Team}[]
[Software Engineering Intern][May 2020 - Aug 2020]

\begin{itemize}
\item Programmed in Python to expose insecure / unpatched internet facing assets to secops for secure configuration
\item Investigated tools and designed an algorithm with scoring model to identify and map assets discovered to enterprise customers
\item Identified internet facing servers belonging to enterprise segment and mapped out secure configurations to reduce risks
\end{itemize}

%% An example of leaving an argument empty
\cvsubsection{GoDaddy: Threat Intelligence \& Detection}[][Software Engineering Intern][May 2019 - Aug 2019]

\begin{itemize}
\item Designed and developed an automated file hash enrichment and investigation process which tripled reports analysed
\item Learned GoLang and understood concepts of concurrency (Goroutines) and templates 
\item Integrated plugins into the Threat Investigation API while maintaining clean interface design
\end{itemize}
 
\section{Projects}
\cvsubsection{Robot Open Autonomous Racing (ROAR™)} [FHL Vive Center
for Enhanced Reality]
[Researcher for a new AI racecar competition hosted in Berkeley][Aug 2019 - Present]
\begin{itemize}
\item Developed and designed the standardized SLAM (localization) platform for all race teams to use 
\item Presentation: http://vivecenter.berkeley.edu/wp-content/uploads/2020/03/Localization.pdf
\item Programming a computer-vision model for ground-plane detection and based on the results identity static and dynamic obstacles 
\item Working under Prof. Allen Yang. Referred by Prof. S. Shankar Sastry
\end{itemize}

\cvsubsection{Computer-Vision Driven Human Robot Interaction} [EECS C106A Final Project]
[Built a functional, cable-actuated robotic hand][Aug 2019 - Dec 2019]
\begin{itemize}
\item Designed and programmed computer vision program to monitor human action (OpenCV and sklearn) 
\item Trained PID controller to ensure accurate detection and classification of action
\item Integrated software (RaspberryPi) with hardware (servo motors) being controlled by Ardiuno to ensure 97\% accurate Interaction
\item Project Website: \href{https://sites.google.com/berkeley.edu/eecs-c106a-final-rps-gp30/}
\end{itemize}

\cvsubsection{Encrypted Contact Sharing through DNS} [GoDaddy Intern Hackathon]
[Used Goaddy tools and infrastructure to add an additional feature to domains][Aug 2019]
\begin{itemize}
\item Designed a double encrypted and signed process to enable domain’s DNS records to exchange contact information
\item Database generated using AWS mySQL and processes defined with AWS Docker
\end{itemize}

\section{Teaching Experience}
\cvsubsection{University of California, Berkeley} []
[EECS C106A/C206A Lab Undergraduate Student Instructor][Aug 2020 - Dec 2020]
\begin{itemize}
\item Robotics I for Undergraduates and Graduate Students
\item Taught by: S. Shankar Sastry
\end{itemize}
%%\cvsubsection{University of California, Berkeley} [] [EECS 16A Lab Assistant] [Jan 2020 - May 2020]
%%\begin{itemize}
%%\item Lab Assistant for Designing Information Devices and Systems II
%%\end{itemize}

\section{Awards}
\textbf{GoDaddy Intern Hackathon - Technical Innovation Award} \hfill \textit{August 2019}\\
\textbf{Cal Alumni Association Leadership Award:} 200 selected for impacting their community \hfill \textit{August 2018 \& August 2019}\\
\textbf{Athena Pinnacle - Qualcomm Award for Computation:} 1 student in San Diego, CA \hfill \textit{May 2018} \\
\textbf{NCWIT - National Honorable Mention:} Top 9\% selected on experience in computing. \hfill \textit{April 2018} \\
\textbf{Rotary Award for Leadership:} Top 3 students selected for impacting their community \hfill \textit{April 2018} 

\end{document}
